\documentclass[12 pt]{scrartcl}
\usepackage{setspace}
\onehalfspacing
\usepackage{amsmath,amssymb,amsfonts,amsthm,mathtools}
\usepackage[english]{babel}
\usepackage[T1]{fontenc}
\usepackage[utf8x]{inputenc}
\usepackage{lmodern}
\usepackage{dsfont}
\usepackage{bbm}
\usepackage[round]{natbib}
\usepackage{color} 
\usepackage[defaultlines=2,all]{nowidow}
\usepackage{caption}
\usepackage[labelformat=simple]{subcaption}
\usepackage{hyperref} % Added for clickable links
\renewcommand\thesubfigure{(\alph{subfigure})}

% Set margin if needed, otherwise scrartcl default is used. 
% Standard academic reports often use geometry, but kept minimal here to match your original style.

\setlength\parindent{0pt}
\setlength{\parskip}{6pt plus 1pt minus 1pt}

\newcommand{\red}{\textcolor{red}}

\begin{document}

\begin{titlepage}
	\centering
	{\scshape\LARGE TU Dortmund \par}
	\vspace{1cm}
	{\scshape\Large On the Theory and Practice of Monte Carlo Simulations \par}
	\vspace{0.5cm}
	{\large Winter Semester 2025/26 \par}
	\vspace{2cm}
	{\huge\bfseries Final Project: [Insert Your Topic Here]\par}
	\vspace{3cm}
	{\Large Author: [Student Name] \par}
	\vspace{0.5 cm}
	{\large Student ID: [Number] \par}
	\vspace{0.5 cm}
	{\Large Group members: [Name 1], [Name 2], [Name 3] (if applicable)}
	\vfill
	{\large Instructors: \par}
	{\large Prof. Dr. Katja Ickstadt \par}
	{\large Zeyu Ding \par}
	\vspace{1cm}
	{\large \today\par}
\end{titlepage}

\tableofcontents
\thispagestyle{empty}

\cleardoublepage

\setcounter{page}{1}

% --- SECTION 1: INTRODUCTION ---
\section{Introduction}

\textcolor{red}{Guidance for this section (approx. 1 page):}
\begin{itemize}
    \item \textcolor{red}{Motivate your chosen topic and clearly state the research question or objective.}
    \item \textcolor{red}{Explain why Monte Carlo methods are specifically appropriate for this problem.}
    \item \textcolor{red}{Provide a brief overview of your approach.}
    \item \textcolor{red}{\textbf{Note on Constraints:} The main report must be between 10 and 15 pages (excluding appendices).}
\end{itemize}

This is your text. Introduce the problem you are solving. For example, if you are working on Option Pricing, explain why analytical solutions might be insufficient and why simulation is needed.

% --- SECTION 2: THEORETICAL BACKGROUND ---
\section{Theoretical Background}

\textcolor{red}{Guidance for this section (approx. 2-3 pages):}
\begin{itemize}
    \item \textcolor{red}{Present relevant theoretical foundations (mathematical derivations, convergence properties, variance analysis, or algorithmic descriptions).}
    \item \textcolor{red}{Demonstrate your understanding of the underlying methodology.}
    \item \textcolor{red}{Avoid simply copying algorithms from slides; provide proper theoretical motivation.}
\end{itemize}

\subsection{Mathematical Derivation}

Present the core math here. For example, the standard Monte Carlo estimator for an integral $I = \int f(x)dx$ is given by:
\[ \hat{I}_n = \frac{1}{n} \sum_{i=1}^n f(X_i) \]
where $X_i$ are i.i.d. samples.

\subsection{Convergence Properties}
Discuss the theoretical convergence rate (e.g., $O(1/\sqrt{n})$) or properties of the Markov Chain if using MCMC.

% --- SECTION 3: DATA DESCRIPTION ---
\section{Data Description}

\textcolor{red}{Guidance for this section (approx. 1 page):}
\begin{itemize}
    \item \textcolor{red}{\textbf{Real-world data:} Document the source (e.g., Kaggle, UCI, Finance data), descriptive statistics, and preprocessing steps.}
    \item \textcolor{red}{\textbf{Simulated data:} Clearly explain the data generating process (DGP), probability distributions, parameter choices, and justify why this design is appropriate.}
\end{itemize}

Describe your dataset here. 

% --- SECTION 4: EMPIRICAL RESULTS ---
\section{Empirical Results}

\textcolor{red}{Guidance for this section:}
\begin{itemize}
    \item \textcolor{red}{Present implementation and results using well-designed figures and tables.}
    \item \textcolor{red}{Conduct convergence diagnostics, sensitivity analyses, or method comparisons.}
    \item \textcolor{red}{Interpret your results in the context of your research question.}
    \item \textcolor{red}{Critical evaluation: Compare different approaches (e.g., Crude MC vs. Variance Reduction, or Frequentist vs. Bayesian).}
\end{itemize}

\subsection{Implementation Details}
Briefly mention the software used (R or Python) and key libraries.

\subsection{Simulation Results}
Figure \ref{fig:convergence} shows the convergence of the estimator.

\begin{figure}[ht]
\centering
% \includegraphics[width=0.7\textwidth]{your_figure_file.pdf} 
% Placeholder box for the example
\framebox[0.7\textwidth]{\rule{0pt}{5cm}Figure Placeholder}
\caption{Convergence of Monte Carlo Estimator as $n$ increases. Ensure axis labels and legends are clear.}
\label{fig:convergence}
\end{figure}

Table \ref{tab:comparison} compares the variance of different methods.

\begin{table}[ht]
\centering
\captionabove{Comparison of Variance Reduction Techniques}
\label{tab:comparison}
\begin{tabular}{l|rr}
Method & Variance & Relative Efficiency \\
\hline
Crude MC & 0.050 & 1.0 \\
Antithetic Variates & 0.012 & 4.2 \\
Control Variates & 0.005 & 10.0
\end{tabular}
\end{table}

% --- SECTION 5: CONCLUSION ---
\section{Conclusion and Discussion}

\textcolor{red}{Guidance for this section (approx. 1 page):}
\begin{itemize}
    \item \textcolor{red}{Summarize your findings.}
    \item \textcolor{red}{Discuss limitations of your study.}
    \item \textcolor{red}{Suggest directions for future work.}
\end{itemize}

This is your text. Summarize the key takeaways.

% --- BIBLIOGRAPHY ---
\newpage
\addcontentsline{toc}{section}{Bibliography}
\renewcommand\refname{Bibliography} 
\bibliographystyle{plainnat}
\bibliography{references} % Make sure you have a references.bib file

% --- APPENDICES ---
\newpage
\appendix 
\addsec{Appendix}

\section*{A \ AI Usage Declaration}
\addcontentsline{toc}{subsection}{A \hspace*{0.15cm} AI Usage Declaration} 

\textcolor{red}{\textbf{Mandatory if AI was used:} You must explicitly disclose the tool and specific nature of its use (e.g., "Used ChatGPT to debug a dimension mismatch error" or "Used Copilot for LaTeX formatting"). If no AI was used, state that the work is entirely your own without AI assistance.}

\emph{Example Statement: I declare that I used ChatGPT-4o to assist with debugging the R code for the Gibbs Sampler and for checking grammatical errors in the Introduction. The derivation of formulas and the analysis of results are my own work.}

\section*{B \ Additional figures}
\addcontentsline{toc}{subsection}{B \hspace*{0.15cm} Additional figures} 

\section*{C \ Additional tables}
\addcontentsline{toc}{subsection}{C \hspace*{0.15cm} Additional tables} 

\section*{D \ Algorithms}
\addcontentsline{toc}{subsection}{D \hspace*{0.15cm} Code Snippets} 
You can include key parts of your algorithmic code here using the listings package if desired.

\end{document}